% !Mode:: "TeX:UTF-8"
%!TEX root = ../main.tex

\markboth{附\quad 录}{附\quad 录}
\addcontentsline{toc}{chapter}{附\quad 录} % 添加到目录中
\chapter*{源程序}
\lstset{language=C}
\begin{lstlisting}
取平均

#include<stdio.h>
#include <stdlib.h>
int main()
{
	FILE *fp1;
	double shuju[180+1][367+1]={0.0};
	double t=0.0;
	double max[180+1]={0};
	int start[180+1]={0};
	int end[180+1]={0};
	FILE *fp2;
	fp2=fopen("C:\\Users\\Dell-pc\\Desktop\\正方形2222.xls","w");
	fprintf(fp2,"shujuzushu\t max\n");
	int i,j=0;
	if(NULL == (fp1 = fopen("C:\\Users\\Dell-pc\\Desktop\\1.txt", "r")))  
    {  
        printf("error\n");  
        exit(1);  
    }
    for(i=1;i<=180;i++)
	{
	 for(j=1;j<=367;j++)
		{
			fscanf(fp1,"%lf",&shuju[i][j]);
			//printf("%lf\n",shuju[i][j]);
		}
		fscanf(fp1,"\n");
	}
	 fclose(fp1);
	
	
	 for(i=1;i<=180;i++)
	{
	
		 for(j=1;j<=367;j++)
		{
			
			
			if(shuju[i][j]==0.0&&shuju[i][j+1]!=0.0)
			{
				start[i]=j+1;
			  }
			  if(shuju[i][j]!=0.0&&shuju[i][j+1]==0.0)
			{
				end[i]=j;
			  }
		}
	}
	for(i=1;i<=180;i++)
	{
		max[i]=(start[i]+end[i])/2;
		fprintf(fp2,"%d\t   %lf\n",i,max[i]);
	}
 }




取最大


#include<stdio.h>
#include <stdlib.h>
int main()
{
	FILE *fp1;
	double shuju[180+1][367+1]={0.0};
	double t=0.0;
	int max[180+1]={0};
	FILE *fp2;
	fp2=fopen("C:\\Users\\Dell-pc\\Desktop\\取最大值222.xls","w");
	fprintf(fp2,"shujuzushu\t max\n");
	int i,j=0;
	if(NULL == (fp1 = fopen("C:\\Users\\Dell-pc\\Desktop\\1.txt", "r")))  
    {  
        printf("error\n");  
        exit(1);  
    }
    for(i=1;i<=180;i++)
	{
	 for(j=1;j<=367;j++)
		{
			fscanf(fp1,"%lf",&shuju[i][j]);
			//printf("%lf\n",shuju[i][j]);
		}
		fscanf(fp1,"\n");
	}
	 fclose(fp1);
	
	
	 for(i=1;i<=180;i++)
	{
		t=0.0; 
		 for(j=1;j<=367;j++)
		{
			if(shuju[i][j]>=t)
			{
				t=shuju[i][j];
				max[i]=j;
			}
		}
	}
	for(i=1;i<=180;i++)
	{
		fprintf(fp2,"%d\t   %d\n",i,max[i]);
	}
 }

CT图像还原


#include<stdio.h>
#include<stdio.h>
#include<stdlib.h>
#include<time.h>
#include<math.h>
#define pi 3.14159265897
#define d 0.278431
double alpha[180+1]={0.0};
double hudu[180+1]={0.0}; 
double k[180+1]={0.0};
double b[180+1]={0.0};
double xiangshu[256+1][256+1]={0.0};
int main()
{
	FILE *fp1;
	FILE *fp2;
	int i,j=0;
	int m,n=0.0;
	double shuju[180+1][512+1]={0.0};
	fp2=fopen("C:\\Users\\Dell-pc\\Desktop\\数据2.xls","w");
	if(NULL == (fp1 = fopen("C:\\Users\\Dell-pc\\Desktop\\新建文本文档.txt", "r")))  
    {  
        printf("error\n");  
        exit(1);  
    }
    for(i=1;i<=180;i++)
	{
	 for(j=1;j<=512;j++)
		{
			fscanf(fp1,"%lf",&shuju[i][j]);
			//printf("%lf\n",shuju[i][j]);
		}
		fscanf(fp1,"\n");
	}
	for(i=1;i<=180;i++)
	{
		alpha[i]=-61.0970+(i-1)*1.001856;
		hudu[i]=(alpha[i]/180)*pi;
		k[i]=tan(hudu[i]);
		b[i]=(56.66592634457987*(256/100)-k[i]*40.8513462978296*(256/100)-(256/100)*256*d/cos(hudu[i]));
	}
	for(i=1;i<=180;i++)
	{
		for(n=1;n<=256;n++)
		{
			for(m=1;m<=256;m++)
			{
				for(j=1;j<=512;j++)
				{
					if(cos(hudu[i])>=0.0)
					{
						if((n>=(k[i]*m+b[i]+(256/100)*(j-1)*d/cos(hudu[i])))&&(n<(k[i]*m+b[i]+(256/100)*j*d/cos(hudu[i]))))
						{
							xiangshu[m][n]=xiangshu[m][n]+shuju[i][j];
						}
					}
					if(cos(hudu[i])<0.0)
					 if((n<=(k[i]*m+b[i]+(256/100)*(j-1)*d/cos(hudu[i])))&&(n>(k[i]*m+b[i]+(256/100)*j*d/cos(hudu[i]))))
						{
							xiangshu[m][n]=xiangshu[m][n]+shuju[i][j];
						}
				
				}
			}
		}
	}
	for(j=256;j>=1;j--)
	{
		for(i=1;i<=256;i++)
		{
			fprintf(fp2,"%lf\t",xiangshu[i][j]);
		}
		fprintf(fp2,"\n");
	}
}
还原吸收率
#include<stdio.h>
#include<stdio.h>
#include<stdlib.h>
#include<time.h>
#include<math.h>
int main()
{
	FILE *fp1;
	int i,j=0; 
	FILE *fp2;
	double shuju[256+1][256+1]={0.0};
	fp2=fopen("C:\\Users\\Dell-pc\\Desktop\\数据23.xls","w");
	if(NULL == (fp1 = fopen("C:\\Users\\Dell-pc\\Desktop\\1.txt", "r")))
    {
        printf("error\n");
        exit(1);
    }
    for(i=1;i<=256;i++)
	{
	 for(j=1;j<=256;j++)
		{
			fscanf(fp1,"%lf",&shuju[i][j]);
			//printf("%lf\n",shuju[i][j]);
		}
		fscanf(fp1,"\n");
	}
	for(i=1;i<=256;i++)
	{
		for(j=1;j<=256;j++)
		{
		
			shuju[i][j]=shuju[i][j]*2.024663305;//变化的
		}
	}
	for(i=1;i<=256;i++)
	{
		for(j=1;j<=256;j++)
		{
			fprintf(fp2,"%lf\t",shuju[i][j]);
		}
		fprintf(fp2,"\n");
	}
}
MATLAB画模板黑白图程序
mycolor = [ 0 0 0
1 1 1];
>> colormap(mycolor)
>> imagesc(unnamed)
>> axis square
>> set(gca,'xtick',[],'xticklabel',[])
set(gca,'ytick',[],'yticklabel',[])

iradon_eccentric.m
function [img,H] = iradon_eccentric(varargin)
%IRADON Inverse Radon transform.
%   I = iradon(R,THETA) reconstructs the image I from projection data in the
%   2-D array R.  The columns of R are parallel beam projection data.
%   IRADON assumes that the center of rotation is the center point of the
%   projections, which is defined as ceil(size(R,1)/2).
%
%   THETA describes the angles (in degrees) at which the projections were
%   taken.  It can be either a vector containing the angles or a scalar
%   specifying D_theta, the incremental angle between projections. If THETA
%   is a vector, it must contain angles with equal spacing between them.  If
%   THETA is a scalar specifying D_theta, the projections are taken at
%   angles THETA = m * D_theta; m = 0,1,2,...,size(R,2)-1.  If the input is
%   the empty matrix ([]), D_theta defaults to 180/size(R,2).
%
%   IRADON uses the filtered backprojection algorithm to perform the inverse
%   Radon transform.  The filter is designed directly in the frequency
%   domain and then multiplied by the FFT of the projections.  The
%   projections are zero-padded to a power of 2 before filtering to prevent
%   spatial domain aliasing and to speed up the FFT.
%
%   I = IRADON(R,THETA,INTERPOLATION,FILTER,FREQUENCY_SCALING,OUTPUT_SIZE)
%   specifies parameters to use in the inverse Radon transform.  You can
%   specify any combination of the last four arguments.  IRADON uses default
%   values for any of these arguments that you omit.
%
%   INTERPOLATION specifies the type of interpolation to use in the
%   backprojection. The default is linear interpolation. Available methods
%   are:
%
%      'nearest' - nearest neighbor interpolation
%      'linear'  - linear interpolation (default)
%      'spline'  - spline interpolation
%      'pchip'   - shape-preserving piecewise cubic interpolation
%      'v5cubic' - the cubic interpolation from MATLAB 5, which does not
%                  extrapolate and uses 'spline' if X is not equally spaced.
%
%   FILTER specifies the filter to use for frequency domain filtering.
%   FILTER is a string that specifies any of the following standard filters:
%
%   'Ram-Lak'     The cropped Ram-Lak or ramp filter (default).  The
%                 frequency response of this filter is |f|.  Because this
%                 filter is sensitive to noise in the projections, one of
%                 the filters listed below may be preferable.
%   'Shepp-Logan' The Shepp-Logan filter multiplies the Ram-Lak filter by
%                 a sinc function.
%   'Cosine'      The cosine filter multiplies the Ram-Lak filter by a
%                 cosine function.
%   'Hamming'     The Hamming filter multiplies the Ram-Lak filter by a
%                 Hamming window.
%   'Hann'        The Hann filter multiplies the Ram-Lak filter by a
%                 Hann window.
%   'none'        No filtering is performed.
%
%   FREQUENCY_SCALING is a scalar in the range (0,1] that modifies the
%   filter by rescaling its frequency axis.  The default is 1.  If
%   FREQUENCY_SCALING is less than 1, the filter is compressed to fit into
%   the frequency range [0,FREQUENCY_SCALING], in normalized frequencies;
%   all frequencies above FREQUENCY_SCALING are set to 0.
%
%   OUTPUT_SIZE is a scalar that specifies the number of rows and columns in
%   the reconstructed image.  If OUTPUT_SIZE is not specified, the size is
%   determined from the length of the projections:
%
%       OUTPUT_SIZE = 2*floor(size(R,1)/(2*sqrt(2)))
%
%   If you specify OUTPUT_SIZE, IRADON reconstructs a smaller or larger
%   portion of the image, but does not change the scaling of the data.
%
%   If the projections were calculated with the RADON function, the
%   reconstructed image may not be the same size as the original image.
%
%   [I,H] = iradon(...) returns the frequency response of the filter in the
%   vector H.
%
%   Class Support
%   -------------
%   R can be double or single. All other numeric input arguments must be double.
%   I has the same class as R. H is double.
%
%   Examples
%   --------
%   Compare filtered and unfiltered backprojection.
%
%       P = phantom(128);
%       R = radon(P,0:179);
%       I1 = iradon(R,0:179);
%       I2 = iradon(R,0:179,'linear','none');
%       subplot(1,3,1), imshow(P), title('Original')
%       subplot(1,3,2), imshow(I1), title('Filtered backprojection')
%       subplot(1,3,3), imshow(I2,[]), title('Unfiltered backprojection')
%
%   Compute the backprojection of a single projection vector. The IRADON
%   syntax does not allow you to do this directly, because if THETA is a
%   scalar it is treated as an increment.  You can accomplish the task by
%   passing in two copies of the projection vector and then dividing the
%   result by 2.
%
%       P = phantom(128);
%       R = radon(P,0:179);
%       r45 = R(:,46);
%       I = iradon([r45 r45], [45 45])/2;
%       imshow(I, [])
%       title('Backprojection from the 45-degree projection')
%
%   See also FAN2PARA, FANBEAM, IFANBEAM, PARA2FAN, PHANTOM, RADON.

%   Copyright 1993-2013 The MathWorks, Inc.

%   References:
%      A. C. Kak, Malcolm Slaney, "Principles of Computerized Tomographic
%      Imaging", IEEE Press 1988.

narginchk(2,6);

[p,theta,filter,d,interp,N] = parse_inputs(varargin{:});

[p,H] = filterProjections(p, filter, d);

% Define the x & y axes for the reconstructed image so that the origin
% (center) is in the spot which RADON would choose.
xleft = -149.1236 + 1;
x = (1:N) - 1 + xleft;
x = repmat(x, N, 1);

ytop = 159.5781 - 1;
y = (N:-1:1).' - N + ytop;
y = repmat(y, 1, N);

len = size(p,1);
ctrIdx = ceil(len/2);     % index of the center of the projections

% Zero pad the projections to size 1+2*ceil(N/sqrt(2)) if this
% quantity is greater than the length of the projections
imgDiag = 2*ceil(N/sqrt(2))+1;  % largest distance through image.
if size(p,1) < imgDiag
    rz = imgDiag - size(p,1);  % how many rows of zeros
    p = [zeros(ceil(rz/2),size(p,2)); p; zeros(floor(rz/2),size(p,2))];
    ctrIdx = ctrIdx+ceil(rz/2);
end

% Backprojection - vectorized in (x,y), looping over theta
switch interp
case'nearest neighbor'

        img = iradonmex(N, theta, x, y, p, 0);

case'linear'

        img = iradonmex(N, theta, x, y, p, 1);

case {'spline','pchip','cubic','v5cubic'}

        interp_method = sprintf('*%s',interp); % Add asterisk to assert
% even-spacing of taxis

% Generate trignometric tables
        costheta = cos(theta);
        sintheta = sin(theta);

% Allocate memory for the image
        img = zeros(N,'like',p);        

for i=1:length(theta)
            proj = p(:,i);
            taxis = (1:size(p,1)) - ctrIdx;
            t = x.*costheta(i) + y.*sintheta(i);
            projContrib = interp1(taxis,proj,t(:),interp_method);
            img = img + reshape(projContrib,N,N);
end

end

img = img*pi/(2*length(theta));

%======================================================================
function [p,H] = filterProjections(p_in, filter, d)

p = p_in;

% Design the filter
len = size(p,1);
H = designFilter(filter, len, d);

if strcmpi(filter, 'none')
return;
end

p(length(H),1)=0;  % Zero pad projections

% In the code below, I continuously reuse the array p so as to
% save memory.  This makes it harder to read, but the comments
% explain what is going on.

p = fft(p);               % p holds fft of projections

p = bsxfun(@times, p, H); % faster than for-loop

p = ifft(p,'symmetric');  % p is the filtered projections

p(len+1:end,:) = [];      % Truncate the filtered projections
%----------------------------------------------------------------------

%======================================================================
function filt = designFilter(filter, len, d)
% Returns the Fourier Transform of the filter which will be
% used to filter the projections
%
% INPUT ARGS:   filter - either the string specifying the filter
%               len    - the length of the projections
%               d      - the fraction of frequencies below the nyquist
%                        which we want to pass
%
% OUTPUT ARGS:  filt   - the filter to use on the projections


order = max(64,2^nextpow2(2*len));

if strcmpi(filter, 'none')
    filt = ones(1, order);
return;
end

% First create a bandlimited ramp filter (Eqn. 61 Chapter 3, Kak and
% Slaney) - go up to the next highest power of 2.

n = 0:(order/2); % 'order' is always even. 
filtImpResp = zeros(1,(order/2)+1); % 'filtImpResp' is the bandlimited ramp's impulse response (values for even n are 0)
filtImpResp(1) = 1/4; % Set the DC term 
filtImpResp(2:2:end) = -1./((pi*n(2:2:end)).^2); % Set the values for odd n
filtImpResp = [filtImpResp filtImpResp(end-1:-1:2)]; 
filt = 2*real(fft(filtImpResp)); 
filt = filt(1:(order/2)+1);

w = 2*pi*(0:size(filt,2)-1)/order;   % frequency axis up to Nyquist

switch filter
case'ram-lak'
% Do nothing
case'shepp-logan'
% be careful not to divide by 0:
        filt(2:end) = filt(2:end) .* (sin(w(2:end)/(2*d))./(w(2:end)/(2*d)));
case'cosine'
        filt(2:end) = filt(2:end) .* cos(w(2:end)/(2*d));
case'hamming'
        filt(2:end) = filt(2:end) .* (.54 + .46 * cos(w(2:end)/d));
case'hann'
        filt(2:end) = filt(2:end) .*(1+cos(w(2:end)./d)) / 2;
otherwise
        error(message('images:iradon:invalidFilter'))
end

filt(w>pi*d) = 0;                      % Crop the frequency response
filt = [filt' ; filt(end-1:-1:2)'];    % Symmetry of the filter
%----------------------------------------------------------------------


%======================================================================
function [p,theta,filter,d,interp,N] = parse_inputs(varargin)
%  Parse the input arguments and retun things
%
%  Inputs:   varargin -   Cell array containing all of the actual inputs
%
%  Outputs:  p        -   Projection data
%            theta    -   the angles at which the projections were taken
%            filter   -   string specifying filter or the actual filter
%            d        -   a scalar specifying normalized freq. at which to crop
%                         the frequency response of the filter
%            interp   -   the type of interpolation to use
%            N        -   The size of the reconstructed image

p     = varargin{1};
theta = varargin{2};

validateattributes(p    ,{'numeric','logical'},{'real','2d'},mfilename,'R'    ,1);
validateattributes(theta,{'numeric','logical'},{'real'}     ,mfilename,'theta',2);

% Convert to radians
theta = pi*theta/180;

% Default values
N = 0;                 % Size of the reconstructed image
d = 1;                 % Defaults to no cropping of filters frequency response
filter = 'ram-lak';    % The ramp filter is the default
interp = 'linear';     % default interpolation is linear

interp_strings = {'nearest neighbor', 'linear', 'spline', 'pchip', 'cubic', 'v5cubic'};
filter_strings = {'ram-lak','shepp-logan','cosine','hamming', 'hann', 'none'};
string_args    = [interp_strings filter_strings];

for i=3:nargin
    arg = varargin{i};
if ischar(arg)
        str = validatestring(arg,string_args,mfilename,'interpolation or filter');
        idx = find(strcmp(str,string_args),1,'first');
if idx <= numel(interp_strings)
% interpolation method
            interp = string_args{idx};
else%if (idx > numel(interp_strings)) && (idx <= numel(string_args))
% filter type
            filter = string_args{idx};
end
elseif numel(arg)==1
if arg <=1
% frequency scale
            validateattributes(arg,{'numeric','logical'},...
                {'positive','real','nonsparse'},mfilename,'frequency_scaling');
            d = arg;
else
% output size
            validateattributes(arg,{'numeric'},{'real','finite','integer'},...
                mfilename,'output_size');
            N = arg;
end
else
        error(message('images:iradon:invalidInputParameters'))
end
end

% If 'cubic' interpolation is specified, convert it to 'pchip'. 
if strcmp(interp,'cubic')
    interp = 'pchip';
end

% If the user didn't specify the size of the reconstruction, so
% deduce it from the length of projections
if N==0
    N = 2*floor( size(p,1)/(2*sqrt(2)) );  % This doesn't always jive with RADON
end

% for empty theta, choose an intelligent default delta-theta
if isempty(theta)
    theta = pi / size(p,2);
end

% If the user passed in delta-theta, build the vector of theta values
if numel(theta)==1
    theta = (0:(size(p,2)-1))* theta;
end

if length(theta) ~= size(p,2)
    error(message('images:iradon:thetaNotMatchingProjectionNumber'))
end
%----------------------------------------------------------------------

filterProjections.m
function [p,H] = filterProjections(p_in, filter, d)

p = p_in;

% Design the filter
len = size(p,1);
H = designFilter(filter, len, d);

if strcmpi(filter, 'none')
return;
end

p(length(H),1)=0;  % Zero pad projections

% In the code below, I continuously reuse the array p so as to
% save memory.  This makes it harder to read, but the comments
% explain what is going on.

p = fft(p);               % p holds fft of projections

p = bsxfun(@times, p, H); % faster than for-loop

p = ifft(p,'symmetric');  % p is the filtered projections

p(len+1:end,:) = [];      % Truncate the filtered projections
%----------------------------------------------------------------------

%======================================================================
function filt = designFilter(filter, len, d)
% Returns the Fourier Transform of the filter which will be
% used to filter the projections
%
% INPUT ARGS:   filter - either the string specifying the filter
%               len    - the length of the projections
%               d      - the fraction of frequencies below the nyquist
%                        which we want to pass
%
% OUTPUT ARGS:  filt   - the filter to use on the projections


order = max(64,2^nextpow2(2*len));

if strcmpi(filter, 'none')
    filt = ones(1, order);
return;
end

% First create a bandlimited ramp filter (Eqn. 61 Chapter 3, Kak and
% Slaney) - go up to the next highest power of 2.

n = 0:(order/2); % 'order' is always even. 
filtImpResp = zeros(1,(order/2)+1); % 'filtImpResp' is the bandlimited ramp's impulse response (values for even n are 0)
filtImpResp(1) = 1/4; % Set the DC term 
filtImpResp(2:2:end) = -1./((pi*n(2:2:end)).^2); % Set the values for odd n
filtImpResp = [filtImpResp filtImpResp(end-1:-1:2)]; 
filt = 2*real(fft(filtImpResp)); 
filt = filt(1:(order/2)+1);

w = 2*pi*(0:size(filt,2)-1)/order;   % frequency axis up to Nyquist

switch filter
case'ram-lak'
% Do nothing
case'shepp-logan'
% be careful not to divide by 0:
        filt(2:end) = filt(2:end) .* (sin(w(2:end)/(2*d))./(w(2:end)/(2*d)));
case'cosine'
        filt(2:end) = filt(2:end) .* cos(w(2:end)/(2*d));
case'hamming'
        filt(2:end) = filt(2:end) .* (.54 + .46 * cos(w(2:end)/d));
case'hann'
        filt(2:end) = filt(2:end) .*(1+cos(w(2:end)./d)) / 2;
otherwise
        error(message('images:iradon:invalidFilter'))
end

filt(w>pi*d) = 0;                      % Crop the frequency response
filt = [filt' ; filt(end-1:-1:2)'];    % Symmetry of the filter
%----------------------------------------------------------------------
Filter.m
clc;
clear all;
a1 = xlsread('C:\path\1.xls');
o1 = filterProjections(a1, 'shepp-logan', 1);
a2 = xlsread('C:\path\2.xls');
o2 = filterProjections(a2, 'shepp-logan', 1);
a3 = xlsread('C:\path\3.xls');
o3 = filterProjections(a3, 'shepp-logan', 1);
subplot(3,2,1),imshow(a1,[]),title('1原');
subplot(3,2,2),imshow(o1,[]),title('1滤波');
subplot(3,2,3),imshow(a2,[]),title('2原');
subplot(3,2,4),imshow(o2,[]),title('2滤波');
subplot(3,2,5),imshow(a3,[]),title('3原');
subplot(3,2,6),imshow(o3,[]),title('3滤波');

image_restruction.m
clc;
clear;
close all;
N = 256; %图像大小
[I] = xlsread('A题附件',1);
theta = -61.0970+90:1.0019:118.2431+90; %投影角度
[K] = xlsread('A题附件',2);
[P] = xlsread('A题附件',3);
[Q] = xlsread('A题附件',5);
rc = iradon_eccentric(K,theta,'linear','Shepp-Logan',1);%直接反投影重建
rec_RL1 = iradon_eccentric(P,theta,'linear','Shepp-Logan',1);
rec_RL2 = iradon_eccentric(Q,theta,'linear','Shepp-Logan',1);
figure;%显示图像
subplot(2,2,1),imshow(I),title('原始图像');
subplot(2,2,2),imshow(rc,[]),title('Problem-1重建图像');
subplot(2,2,3),imshow(rec_RL1,[]),title('Problem-1重建图像');
subplot(2,2,4),imshow(rec_RL2,[]),title('Problem-1重建图像');
%imshow(rc,[]),title('直接反投影重建图像');


build.m
clc;
clear ;
close all;
theta = 0:1:179;
or1 = xlsread('C:\path\or1.xls','sheet1');
[t1,xp1] = radon(or1,theta);
xlswrite('C:\path\output.xls',t1,'sheet1');
or2 = xlsread('C:\path\or1.xls','sheet2');
[t2,xp2] = radon(or2,theta);
xlswrite('C:\path\output.xls',t2,'sheet2');
or3 = xlsread('C:\path\or1.xls','sheet3');
[t3,xp3] = radon(or3,theta);
xlswrite('C:\path\output.xls',t3,'sheet3');


\end{lstlisting}