% !Mode:: "TeX:UTF-8"
%!TEX root = ../main.tex

% \markboth{附\quad 录}{附\quad 录}
% \addcontentsline{toc}{chapter}{附\quad 录} % 添加到目录中

\chapter*{源程序}
\lstset{language=C}
\begin{lstlisting}

#include<stdio.h>
#include <stdlib.h>
int main()
{
	FILE *fp1;
	double shuju[180+1][367+1]={0.0};
	double t=0.0;
	double max[180+1]={0};
	int start[180+1]={0};
	int end[180+1]={0};
	FILE *fp2;
	fp2=fopen("C:\\Users\\Dell-pc\\Desktop\\正方形2222.xls","w");
	fprintf(fp2,"shujuzushu\t max\n");
	int i,j=0;
	if(NULL == (fp1 = fopen("C:\\Users\\Dell-pc\\Desktop\\1.txt", "r")))  
    {  
        printf("error\n");  
        exit(1);  
    }
    for(i=1;i<=180;i++)
	{
	 for(j=1;j<=367;j++)
		{
			fscanf(fp1,"%lf",&shuju[i][j]);
			//printf("%lf\n",shuju[i][j]);
		}
		fscanf(fp1,"\n");
	}
	 fclose(fp1);
	
	
	 for(i=1;i<=180;i++)
	{
	
		 for(j=1;j<=367;j++)
		{
			
			
			if(shuju[i][j]==0.0&&shuju[i][j+1]!=0.0)
			{
				start[i]=j+1;
			  }
			  if(shuju[i][j]!=0.0&&shuju[i][j+1]==0.0)
			{
				end[i]=j;
			  }
		}
	}
	for(i=1;i<=180;i++)
	{
		max[i]=(start[i]+end[i])/2;
		fprintf(fp2,"%d\t   %lf\n",i,max[i]);
	}
 }
\end{lstlisting}